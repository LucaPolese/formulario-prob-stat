\documentclass[10pt,a4paper]{article}
\usepackage[utf8]{inputenc}
\usepackage[margin=0.3cm]{geometry}
\usepackage{amssymb}
\usepackage{amsmath}
\renewcommand{\baselinestretch}{0.74}
\usepackage{multicol}
\usepackage{color}
\usepackage{bbold}
\usepackage{relsize}
\usepackage{multicol}
\usepackage{geometry}
\usepackage{graphicx}
\usepackage{makecell}
\usepackage{enumitem}

\newcommand{\indep}{\rotatebox[origin=c]{90}{$\models$}}
\setitemize{noitemsep,topsep=2.5pt,parsep=2.5pt,partopsep=2.5pt}

\begin{document}
\pagenumbering{gobble}
\section*{Formulario di probabilità e statistica}
\setlength{\columnseprule}{0.5pt}
\def\columnseprulecolor{\color{black}}

\begin{multicols}{2}
\setlength{\columnsep}{0.5cm}

\subsection*{Insiemistica}
Se $E$ e $F$ sono tali che $E \cap F= \varnothing $ allora $E$ e $F$ sono incompatibili o disgiunti, non si possono verificare contemporaneamente\\\\
Leggi di De Morgan:
\begin{itemize}
	\item $(A \cap B)^c = A^c \cup B^c$
	\item $(A \cup B)^c = A^c \cap B^c$
\end{itemize}
Disgiunzioni: siano $F, E \in \Omega$ e $\{E,E^c\}$ partizione di $\Omega$
\begin{itemize}
    \item Una partizione di $\Omega$ è composta da eventi a due a due disgiunti e la cui unione forma $\Omega$
    \item $F=(F \cap E) \cup (F \cap E^c)$
    \item $E \cup F = (E \setminus F) \cup (E \cap F) \cup (F \setminus E)$
    \item $E \cup F = E \cup (F \setminus E)$
\end{itemize}
Formula di inclusione-esclusione: \\
$P(E \cup F) = P(E) + P(F) - P(E \cap F)$\\\\
Proprietà misura di probabilità: $P(E)=1-P(E^c)$
\subsection*{Conteggio}
\subsubsection*{Disposizioni}
Dati $n$ elementi, il numero di sottinsiemi formati da $k$ elementi, in cui due sottinsiemi differiscano tra loro per almeno un elemento o per l'ordine
\begin{itemize}
	\item Disposizioni con ripetizione: disposizioni in cui uno stesso elemento può comparire fino a $k$ volte nello stesso gruppo: $n\cdot n\cdot n\cdot...\cdot n = n^k$.
	\item Disposizioni senza ripetizione: disposizioni in cui in ogni sottinsieme i $k$ elementi sono tutti distinti tra loro: $ n \cdot (n-1)\cdot (n-2) \cdot ... \cdot (n-k+1)$ con $k$ fattori.
	\item Permutazioni: disposizioni senza ripetizione in cui $n=k$: $n!$
\end{itemize}
\subsubsection*{Combinazioni}
Dati $n$ elementi, il numero di sottinsiemi formati da $k$ elementi, in cui due sottinsiemi differiscano tra loro per almeno un elemento e non per l'ordine, e uno stesso elemento può comparire fino a $k$ volte nello stesso gruppo: ${n \choose k} = \frac{n!}{k!\cdot (n-k)!}$
\subsection*{Probabilità condizionata}
Probabilità condizionata: 
\begin{itemize}
	\item Formula di moltiplicazione: $P(E \cap F) = P(E|F) \cdot P(F)$
	\item $P(E^c|F) = 1-P(E|F)$
	\item $P(E\cup G | F) = P(E|F) + P(G|F) - P(E \cap G |F)$
\end{itemize}
Formula delle probabilità totali: $P(E)=P(E|F) \cdot P(F) + P(E|F^c) \cdot P(F^c)$\\\\
Formula di Bayes: $P(F|E) = \frac{P(E|F) \cdot P(F)}{P(E)}$ 
\subsection*{Eventi indipendenti}
Due eventi $E,F$ si dicono indipendenti se $P(E \cap F) = P(E) \cdot P(F)$. Inoltre:
\begin{itemize}
	\item $\{E,F\}$ indipendenti $\iff P(E|F)=P(E) \iff P(F|E)=P(F)$
	\item $\{E,F\}$ indipendenti $\iff \{E^c,F\}$ indipendenti $\iff \{E^c,F^c\}$ indipendenti $\iff \{E,F^c\}$ indipendenti
\end{itemize}

\subsection*{Variabili aleatorie}
Formula valor medio: $E(X) = \sum\limits_{x_k \in \chi} x_k \cdot p_x(x_k)$\\
Proprietà valor medio:
\begin{itemize}
\item $E(aX) = a\cdot E(x)$
\item $E(X+Y) = E(X) + E(Y)$
\item $X \geq Y \Rightarrow E(X) \geq E(Y)$
\end{itemize}
Valor medio:
$E(g(X)) = \sum\limits_{x_k \in \chi} (g(x_k)\cdot p_x(x_k))$\\\\
Varianza: $Var(X) = \sum\limits_{x_k \in \chi}((x_k-E(X))^2\cdot p_x(x_k))$\\

\subsection*{Varianza}
\textit{Sia $X$ v.a. con legge $P_X$ e alfabeto composto da elementi $x_k$. Si definisce la varianza:}

$Var(X) = \mathlarger{\sum_{x_k}}[x_k - E(X)]^2 P_X(x_k)$

\textit{Valgono:}
\begin{itemize}
\item $Var(X) = E[(X-E(X))^2] = E(X^2) - E(X)^2$
\item $Var(X) \geq 0$, in particolare $Var(X) = 0 \Leftrightarrow X \equiv costante$
\item $Var(aX) = a^2 Var(X)$, $a \in \mathbb{R}$
\item $Var(X + a) = Var(X)$, $a \in \mathbb{R}$
\item $Var(X+Y)=Var(X)+Var(Y)+2Cov(X,Y)$
\item $X \indep Y \Longrightarrow Var(X+Y) = Var(X) + Var(Y)$
\end{itemize}
% covarianza
\textbf{Def} Covarianza. \textit{Siano $X,\,Y$ v.a.. Si definisce la covarianza:}

$Cov(X, Y) = E[(X - E(X))(Y - E(Y))]$

\textit{Valgono:}
\begin{itemize}
\item $Cov(X, Y) = E(XY) - E(X)E(Y)$
\item $Cov(Y, X) = Cov(X, Y)$
\item $Cov(aX + b, Y) = aCov(X, Y)$
\item $Cov(X + Y, Z) = Cov(X, Z) + Cov(Y, Z)$
\item $X \indep Y \Longrightarrow Cov(X, Y) = 0$
\end{itemize}
% Variabili scorrelate
\textbf{Def} Indipendenza. $X \indep Y$ se:
$P_{XY}(x_i, y_j) = P_X(x_i)P_Y(y_j) \quad\forall x_i, y_j$
\begin{itemize}
\item $X \indep Y \Longrightarrow Var(X+Y) = Var(X) + Var(Y)$
\item $X \indep Y \Longrightarrow Var(aX+bY+c) = a^2 Var(X) + b^2 Var(Y)$
\end{itemize}
\textbf{Def} V.a. scorrelate. \textit{Le v.a. $X$, $Y$ sono dette scorrelate se $Cov(X, Y) = 0$}\\
\begin{itemize}
    \item Indipendenza $\Longrightarrow$ Scorrelazione
    \item Non valido il contrario
\end{itemize}


\subsection*{Variabili aleatorie discrete notevoli}
\subsubsection*{Alfabeto finito: Variabili aleatorie di Bernoulli}
\begin{itemize}
    \item $X\sim Be(p)$ con $p \in [0, 1]$ 
    \item Alfabeto $\mathcal{X} = \{0,1\}$
    \item Densità discreta $p_x(1) = p$, $p_x(0) = 1-p$
    \item $E(X) = p \qquad Var(X) = p(1-p)$
\end{itemize}
\subsubsection*{Alfabeto finito: Variabili aleatorie binomiali}
Conta il numero di successi in uno schema di Bernoulli con $n$ prove indipendenti, dove la probabilità di successo è $p$
\begin{itemize}
    \item $X\sim Bin(n,p)$ con $n \in \mathbb{N} = \{1,2,..\}$ e $p \in [0, 1]$ 
    \item Alfabeto $\mathcal{X} = \{0,1,..,n\}$
    \item Densità discreta $p_x(k) ={n \choose k} \cdot p^k \cdot (1-p)^{n-k}$, $k \in \mathcal{X}$
    \item Inoltre $E(X) = np \qquad Var(X) = n\cdot p\cdot(1-p)$
\end{itemize}
\subsubsection*{Alfabeto infinito: Variabili aleatorie geometriche}
Corrisponde al numero di prove che devo effettuare per osservare il primo successo in uno schema di Bernoulli dove il numero di prove non è necessariamente predefinito e dove la probabilità di successo è $p$
\begin{itemize}
    \item $X\sim Ge(p)$ con $p \in (0, 1)$ 
    \item Alfabeto $\mathcal{X} = \mathbb{N} = \{1,2,..\}$
    \item Densità discreta $p_x(k) =(1-p)^{k-1}\cdot p$, $k \in \mathcal{X}$
    \item Inoltre $E(X) = \frac{1}{p} \qquad Var(X) = \dfrac{1-p}{p^2}$
\end{itemize}
Variante: invece di guardare alla prova che corrisponde al primo successo si guarda il numero di fallimenti prima del primo successo:
\begin{itemize}
    \item $X'=X-1, \, X\sim Ge(p)$ con $p \in (0, 1)$ 
    \item Densità discreta $p_x(k) =(1-p)^{k}\cdot p$, $k \in \mathcal{X}$
    \item Inoltre $E(X) = \frac{1-p}{p} \qquad Var(X) = \dfrac{1-p}{p^2}$
\end{itemize}
Probabilità di lunga attesa: indica la probabilità che per il primo successo si debba aspettare più di qualcosa: $P(X > k) = (1-p)^k$ 
\subsubsection*{Alfabeto infinito: Variabili aleatorie di Poisson}
Una v.al. Binomiale con $n$ molto grande e $p$ molto piccolo può essere approssimata da una v.al di Poisson di parametro $\lambda = n\cdot p$.
Euristica: $n > 100;\; p < 0.01;\; np \leq 20$
\begin{itemize}
    \item $X\sim Po(\lambda)$ con $\lambda > 0$ 
    \item Alfabeto $\mathcal{X} = \mathbb{N}_0 = \{0,1,..\}$
    \item Densità discreta $p_x(k) =e^{-\lambda}\cdot\frac{\lambda^k}{k!}$, $k \in \mathcal{X}$
    \item Inoltre $E(X) = \lambda \qquad Var(X) = \lambda$
\end{itemize}


\subsection*{Vettori aleatori discreti}
Siano X v.al con alfabeto $\mathcal{X}=\{x_1,x_2,..\}$ e Y v.al con alfabeto $ \mathcal{Y}=\{y_1,y_2,..\}$ allora per l'alfabeto $\mathcal{V}$ del vettore aleatorio $\underline{v}$ si ha $\mathcal{V}\subseteq\mathcal{X}\times\mathcal{Y}$
\begin{itemize}
\item Densità congiunta: $p_{XY}(x_i, y_j) = P(X=x_i, Y=y_j)$
\item Densità marginale X: $p_X(x_i) = \sum\limits_{y_j \in \mathcal{Y}} p_{XY}(x_i,y_j) \; \forall x_i \in \mathcal{X}$
\item Densità marginale Y: $p_Y(y_j) = \sum\limits_{x_i \in \mathcal{X}} p_{XY}(x_i,y_j) \; \forall y_j \in \mathcal{Y}$
\item Valor medio: $E[g(XY)] = \sum\limits_{x_i, y_j} g(x_i, y_j) p_{XY}(x_i, y_j)$
\item Valor medio: $E(XY) = \sum\limits_{x_i \in \mathcal{X}} \sum\limits_{y_j \in \mathcal{Y}} x_i y_j P(X=x_i,Y=y_j)$
\end{itemize}
Siano $X$, $Y$ v.al. con alfabeti composti da elementi $x_i$ e $y_j$. Allora X e Y si dicono indipendenti se: $P_{XY}(x_i, y_j) = P_X(x_i)P_Y(y_j) \quad\forall x_i, y_j$


\subsection*{Variabili aleatorie assolutamente continue}

\textbf{Def}
    Una v.a. $X$ si dice (ass.) continua si definisce associando una densit\`a $f_x : \mathbb{R}\to\mathbb{R}$ tale che:
\begin{itemize}
    \item $f_X(x) \geq 0$
    \item $\int_\mathbb{R} f_X(x) dx = 1$
\end{itemize}
Valgono:
\begin{itemize}
    \item $P(X \in I = P_X(I) = \int_I f_X(x)dx$
    \item $P_X(a) = 0\quad\forall a\in\mathbb{R}$
\end{itemize}
\textbf{Def} Funzione di distribuzione. \\  \textit{
    $F_X : \mathbb{R} \to [0, 1]$ \\
    $x \mapsto F_X(x) = P(X \leq x)$
}\\
\textbf{Def} Valor medio v.a.c.. \textit{
    $E(X) = \mathlarger{\int_\mathbb{R}} x f_X(x)dx$
}\\
\textbf{Def} Varianza v.a.c.. \textit{
    $Var(X) = \mathlarger{\int_\mathbb{R}} [x - E(X)]^2 f_X(x)dx$
}\\
\textbf{Def} Valor medio di una funzione $g(X)$, con $X$ avente densità $f_X$:.. \textit{
    $E(g(X)) = \mathlarger{\int_\mathbb{R}} g(x) f_X(x)dx$
}\\








\subsection*{Variabili aleatorie continue notevoli}
\subsubsection*{Variabili aleatorie uniformi}
$X\sim U(a,b)$ con $a<b$ se ha:
\begin{itemize}
    \item densità $f_x(x) =$
        $ \begin{cases}
            \frac{1}{b-a} &\mbox{se } a\leq x\leq b \\
            0 &\mbox{altrimenti}
        \end{cases} $
    \item distribuzione $F_x(x) =$
        $ \begin{cases}
            0 &\mbox{se } x<a\\
            \frac{x-a}{b-a} &\mbox{se } a\leq x\leq b \\
            1 &\mbox{se } x>b
        \end{cases} $
    \item $E(X)=\frac{b+a}{2} \qquad Var(X)=\frac{(b-a)^2}{12}$
\end{itemize}
    
    
    
\subsubsection*{Variabili aleatorie esponenziali}
    Descrive la durata della vita di un fenomeno privo di memoria. Valgono:
    $X \sim Exp(\lambda)$ con densit\`a:
    $f_X(x)$ = $ \begin{cases}
        \lambda e^{-\lambda x} & \text{se } x \geq 0 \\
        0 & \text{altrimenti}
    \end{cases} $\\
    Funzione di distribuzione:
    $F_X(x)$ = $ \begin{cases}
        1-e^{-\lambda x} & \text{se } x \geq 0 \\
        0 & \text{altrimenti}
    \end{cases} $
    
    \begin{itemize}
        \item $E(X) = \frac{1}{\lambda}$
        \item $Var(X) = \frac{1}{\lambda^2}$
        \item $P(X > T+t \mid X > T) = P(X > t)$
    \end{itemize}


\subsubsection*{Variabili aleatorie gaussiane(o normali)}
    $X \sim N(\mu, \sigma^2)$ e densità:
    $f_X(x) = \frac{1}{\sqrt{2\pi\sigma^2}} \exp\left[-\frac{(x-\mu)^2}{2\sigma^2}\right]$
    $\sigma^2$=Varianza\\
    $\sigma$=Deviazione standard\\
    \begin{itemize}
    \item $E(X) = \mu$
    \item $Var(X) = \sigma^2$
    \item $P(Z \leq x) = \Phi(x)$
    \item $P(|Z| \leq x) = 2\Phi(x)-1$
    \item $P(Z \geq x) = 1-\Phi(x) = \Phi(-x)$
    \item $P(|Z| \geq x) = 2(1-\Phi(x))=2\Phi(-x)$
    \item $P(x \le Z \le y) = \Phi(y)-\Phi(x)$
    \item $P(\mu - 4\sigma \leq X \leq \mu + 4\sigma) \approx 1$
    \end{itemize}    
\textbf{Def} Gaussiana standard. \textit{ 
    $X \sim N(0,1)$.
}\\
\textbf{Prop} Trasformazioni affini di v.a. normali. \textit{
    Sia $X \sim N(\mu, \sigma^2)$ e $Y=aX+b$ allora
    $Y \sim N(a\mu+b, a^2\sigma^2)$\\
    quindi se $X \sim N(\mu, \sigma^2) \Rightarrow$
    $Z = \frac{X-\mu}{\sigma} \sim N(0, 1)$
}\\
Se due v.al. $X,Y$ sono indipendenti e $X \sim ~N(\mu_1, \sigma_1^2), \ Y \sim ~N(\mu_2, \sigma_2^2), \ Z=X+Y$ allora:\\
$Z \sim N(\mu_1 + \mu_2, \sigma_1^2 + \sigma_2^2)$

\subsection*{Teoremi Limite}

\textbf{Def} Legge dei grandi numeri (LLN). \textit{
Siano $X_1, \dots, X_n$ v.a. i.i.d. con media finita. Allora
$\overline{X_n} \to E(X_1)$
}\\\\
\textbf{Def} Metodo Monte Carlo. \textit{
$\mathcal{I} = \int^b_a f(x)dx \quad b>a$\\
$\mathcal{I} = (b-a)\int^b_a\frac{f(x)}{b-a} dx = (b-a)E(f(x))$
con $X = U(a, b)$
}\\\\
\textbf{Def} Teorema centrale del limite (CLT). \textit{
Siano $X_1, \dots, X_n$ v.a. i.i.d. con $E(X_1) = \mu$, $Var(X_1) = \sigma^2$. Allora:\\
  $\sqrt{n} \left( \frac{\overline{X_n}-\mu}{\sigma} \right) \to N(0,1)$ \\
  $\frac{\sum\limits_{i=1}^n (X_i - \mu)}{\sigma\sqrt{n}} \to N(0,1)$ \\
  $\sqrt{n}(\overline{X_n}-\mu) \to N(0,\sigma^2)$
}

\subsection*{Statistica Descrittiva}

\subsubsection*{Indici di posizione}
Media campionaria:
$\overline{x}=\frac{1}{n} \sum_{i=1}^n x_i$\\
Mediana:
$M=$ 
$\begin{cases}
    x_{(n+1)/2} & \text{se } n \text{ dispari} \\
    \frac{1}{2} \left( x_{n/2} + x_{(n/2)+1}\right) & \text{se } n \text{ pari} \\
\end{cases} $ \\
Moda: dato a cui corrisponde la frequenza assoluta massima

\subsubsection*{Indici di dispersione}

Varianza campionaria:
$s^2=\dfrac{1}{n-1} \sum_{i=1}^n (x_i-\overline{x})^2$\\
$p$-esimo quantile ($0<p<1$): \\ $q_p=$
$\begin{cases}
    x_{\lfloor np \rfloor + 1 } & \text{se } np \text{ non è intero} \\
    \frac{1}{2} \left( x_{np} + x_{np + 1}\right) & \text{se } np \text{ è intero} \\
\end{cases} $ \\
con $n$ ampiezza del campione
\begin{itemize}
    \item 25-esimo percentile=$Q_1$=primo quartile = $q_{0,25}$
    \item 50-esimo percentile=$Q_2$=secondo quartile (mediana) = $q_{0,5}$
    \item 75-esimo percentile=$Q_3$=terzo quartile = $q_{0,75}$
\end{itemize}
Differenza interquantile: $IQR=Q_3-Q_1$\\
Boxplot: 
\begin{itemize}
    \item $IQR$
    \item Limite del baffo inferiore $= L = Q_1-1.5 \cdot  IQR$
    \item Limite del baffo superiore $= U = Q_3+1.5 \cdot  IQR$
    \item Outliers: dati che rimangono fuori dai baffi
\end{itemize}
 
\subsection*{Statistica Inferenziale}

\subsubsection*{Stimatori}
Uno stimatore $T$ si dice corretto/non distorto se $E_{\underline{\theta}} (T) = \tau (\underline{\theta})$\\
Uno stimatore $T$ si dice consistente se $\text{Var}_{\underline{\theta}}(T) \overset{n\to +\infty}{\longrightarrow 0}$


\end{multicols}
\end{document}